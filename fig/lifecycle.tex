\begin{figure}
  \centering
\begin{columns}
\begin{column}{0.20\textwidth}
  \centering
    \includegraphics[width=\textwidth,trim={300 300 250 300},clip]{img/lifecycle-1}\\
    \footnotesize \textbf{(a)} unicellular group
\end{column}
\begin{column}{0.05\textwidth}
  \vspace{3ex}
  \includegraphics[width=\textwidth]{img/arrow}
\end{column}
\begin{column}{0.20\textwidth}
  \centering
    \includegraphics[width=\textwidth,trim={300 300 250 300},clip]{img/lifecycle-2}\\
    \footnotesize \textbf{(b)} multicellular group
\end{column}
\begin{column}{0.05\textwidth}
  \vspace{3ex}
  \includegraphics[width=\textwidth]{img/arrow}
\end{column}
\begin{column}{0.20\textwidth}
  \centering
    \includegraphics[width=\textwidth,trim={300 300 250 300},clip]{img/lifecycle-3}\\
    \footnotesize \textbf{(c)} multicell w/ propagule
\end{column}
\begin{column}{0.2\textwidth}
\caption{
A single cell (a) begets a multicellular resource-collecting group (b) by cellular reproduction.
A cell group may beget a propagule group (c), shown by change in color.
%\cite{moreno
}
\label{fig:lifecycle}
\end{column}
\end{columns}
\end{figure}
